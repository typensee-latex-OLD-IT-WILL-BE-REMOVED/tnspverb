\documentclass[12pt,a4paper]{article}

\makeatletter
	\input{../config/header[fr].sty}
	
	\usepackage{01-env}
\makeatother


\begin{document}

\section{Écriture \texttt{pseudo verbatim}}

\newparaexample{Avec toutes les options}

\inputlatexex{examples/pseudo-verb-all-options.extra.tex}


Voici comment s'utilise chacune des clés.

\begin{enumerate}
	\item \verb#title# permet de donner un titre si un cadre est utilisé. Un titre vide sera ignoré. Par défaut cette option est de valeur vide.

	\item \verb#frame# demande d'ajouter un cadre autour du contenu.

	\item \verb#center# sert à centrer le contenu.

	\item \verb#width# permet de donner un coefficient multiplicatif à appliquer à la largeur de la ligne. Par défaut cette option vaut \verb#1#.
\end{enumerate}


% ---------------------- %


\newparaexample{Juste un cadre en plus}

\inputlatexex{examples/pseudo-verb-just-frame.extra.tex}


% ---------------------- %


\newparaexample{Sans option}

Dans l'exemple suivant, il ne faut pas oublier d'utiliser \verb#[]# avec l'environnement.
 
\inputlatexex{examples/pseudo-verb-no-option.extra.tex}


Voici ce qu'il se passe si l'on oublie \verb#[]# avant le contenu
\footnote{
	\LaTeX\ ignore les espaces au début du contenu car il commence par chercher un crochet ouvrant indiquant une option du point de vue de l'environnement.
}.
 
\inputlatexex{examples/pseudo-verb-no-option-bad.extra.tex}


% ---------------------- %


\newparaexample{Macros interprétées}

Ci-après est utilisée \macro{squaremacro} qui a été définie par \verb#\newcommand\squaremacro{$x^2$}#.

\newcommand\squaremacro{$x^2$}

\inputlatexex{examples/pseudo-verb-env-with-macro.extra.tex}


\begin{remark}
	Ceci permet de définir des contenus indentés de type langage de programmation assez facilement via la définition de macros de mise en forme de mots clé.
\end{remark}


% ---------------------- %


\newparaexample{Du contenu sur plusieurs pages}

Même avec un cadre un contenu pourra se trouver sur des pages successives.

\begin{figure}[hbt!]
	\centering
	\frame{\includegraphics[scale = .5]{images/pseudo-verb-broken[fr].png}}
\end{figure}


% ---------------------- %


\section{Fiches techniques}

\IDenv[o]{pseudoverb}{1}

\IDoption{} on utilise un système clé/valeur.
\begin{enumerate}
	\item \verb#title# est le titre du cadre.
	      Par défaut \verb#title = {}#.

	\item \verb#frame# demande d'ajouter un cadre.

	\item \verb#center# sert à centrer le contenu.

	\item \verb#width# donne le coefficient multiplicatif à appliquer à la largeur de la ligne.
	      Par défaut \verb#width = 1#.
\end{enumerate}

\end{document}
